\documentclass[11pt,a4paper,BCOR12mm, headexclude, footexclude, twoside, openright]{scrartcl} 
\usepackage[scaled]{helvet}
\usepackage[british]{babel}
\usepackage[utf8]{inputenc}
\usepackage[T1]{fontenc}
\usepackage{fancyhdr}
\usepackage{lastpage}
\usepackage{ifthen}
\usepackage{amsmath,amsfonts,amsthm}
\usepackage{sfmath}
\usepackage{makecell}
\usepackage{booktabs}
\usepackage{sectsty}

%\KOMAoptions{optionenliste}
%\KOMAoptions{Option}{Werteliste}


\addtokomafont{caption}{\small}
%\setkomafont{descriptionlabel}{\normalfont
%	\bfseries}
\setkomafont{captionlabel}{\normalfont
	\bfseries}
\let\oldtabular\tabular
\renewcommand{\tabular}{\sffamily\oldtabular}
\KOMAoptions{abstract=true}
%\setkomafont{footnote}{\sffamily}
%\KOMAoptions{twoside=true}
%\KOMAoptions{headsepline=true}
%\KOMAoptions{footsepline=true}
\renewcommand\familydefault{\sfdefault}
\renewcommand{\arraystretch}{1.1}
\newcommand{\horrule}[1]{\rule{\linewidth}{#1}}
\setlength{\textheight}{230mm}
\allsectionsfont{\centering \normalfont\scshape}
\let\tmp\oddsidemargin
\let\oddsidemargin\evensidemargin
\let\evensidemargin\tmp
\reversemarginpar

\numberwithin{equation}{section} % Number equations within sections (i.e. 1.1, 1.2, 2.1, 2.2 instead of 1, 2, 3, 4)
\numberwithin{figure}{section} % Number figures within sections (i.e. 1.1, 1.2, 2.1, 2.2 instead of 1, 2, 3, 4)
\numberwithin{table}{section} % Number tables within sections (i.e. 1.1, 1.2, 2.1, 2.2 instead of 1, 2, 3, 4)

\setlength\parindent{0pt}


\begin{document}
%\sffamily
\fancypagestyle{plain}
{%
  \renewcommand{\headrulewidth}{0pt}%
  \renewcommand{\footrulewidth}{0.5pt}
  \fancyhf{}%
  \fancyfoot[R]{\emph{\footnotesize Page \thepage\ of \pageref{LastPage}}}%
  \fancyfoot[C]{\emph{\footnotesize Samy Aittahar}}%
}

\thispagestyle{plain}

\titlehead
{
	University of Liège\hfill
    INFO0049-1%
}
\subject{\vspace{-1ex} \horrule{2pt}\\[0.15cm] {\textsc{\texttt{Knowledge Representation}}}}
\title{Assignment 1}
\subtitle{\textsc{\texttt{Prolog Genesis : Tree searches and lists}}\\\horrule{2pt}\\[0.5cm]}
\author{\bfseries{Samy Aittahar}\vspace{-2ex}}
\date{\begin{tabular}{cc}
  \textsc{Date:}& \textsc{\emph{\today}}
\end{tabular}}
\maketitle

%\begin{abstract}
%\end{abstract}

%--------------------------------------------

\section{Pratical Informations}
\begin{itemize}
	\item Teaching Assistant : Samy Aittahar ;
    \item saittahar@uliege.ac.be, office I-?? (Montefiore Institute, 1st floor) ;
    \item Format : First course will consist on short talk, interactive exercises and homework. Following will be similar, without exercises.

\end{itemize}


\section{Drawing some search trees} %add a * after \section to get rid of the numbering

Given the program in \ref{acpp}, draw the search tree for the following queries : 
\begin{itemize}
	\item has\_killed(X,Y), father(Y,X).
    \item templar(X), has\_killed(Y,X). 
\end{itemize}

Draw also the search trees for the following predicates :

\begin{itemize}
\item assassin(Y), has\_killed(Y,X).
\item has\_killed(Y,X), assassin(Y).
\end{itemize}

What can you tell?

\section{Homework}

	\begin{itemize}
    	\item Write the predicate concat(+L1,+L2,-L3) which succeeds if L3 is the concatenation of lists L1 and L2 (example : concatenation of [a,b] and [c,d] is [a,b,c,d])
    	\item Write the predicate flatten(+L,-L2) which succeeds if L2 is the flat version of L. You should use the concat predicate inside.
        \begin{itemize}
        	\item Be careful about the input/output specifications
            \item A flat list of a list contains only the atoms of the latter. For example, [a,b,c,d] is the flatten version of [a,[b,c,[d]]]
        \end{itemize}
        \item From your predicates, draw the search tree of flatten([[[a]], b],L2).
    \end{itemize}
    
    Homework is expected to be send by email before the 28th of February. Correction will be interactively provided during the following course.

\appendix



\section{Program}
\label{acpp}
\begin{enumerate}
\item        assassin(desmund). ;
\item        assassin(william). ;
\item        assassin(connor). ;
\item        assassin(achilles). ;
\item        assassin(ezio). ;
\item        assassin(altair). ;
\item        templar(haytham). ;
\item        templar(charles). ;
\item        templar(vidic). ;
\item       templar(cesare). ;
\item       has\_killed(desmund, vidic). ;
\item       has\_killed(connor, haytham). ;
\item       has\_killed(ezio, cesare). ;
\item       father(william, desmund). ;
\item       father(haytham, connor). .
\end{enumerate}
%\begin{description}
%	\item[empty] ist der Seitenstil, bei dem Kopf- und Fusszeile vollständig 	leer bleiben.\marginpar[\textit{Randnotiz links}]
%	{\textit{Randnotiz rechts}}
%    \item[plain] ist der Seitenstil, bei dem keinerlei Kolumnentitel verwendet wird.
%    \item[headings] ist der Seitenstil für automatische Kolumnentitel.
%    \item[myheadings] ist der Seitenstil für manuelle Kolumnentitel.
%\end{description}



%$\mathsf{({m}^{3}/s)}$
%-------------------------------
%\begin{figure}
%	\setcapindent{-1em}
%    \fbox{\parbox{.95\linewidth}{
%    	\centering\KOMAScript}}
%	\caption{Beispiel mit teilweise hängendem Einzug ab der zweiten Zeile}
%\end{figure}

%Guten Morgen\footnote{irgend ein text\label{refnote}}\multiplefootnoteseparator\footnote{es geht noch weiter}Blabla\footref{refnote}
\end{document}